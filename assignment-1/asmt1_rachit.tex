\documentclass{article}
\usepackage[utf8]{inputenc}
\usepackage{amsmath}
\usepackage{algorithm}
\usepackage{algpseudocode}

\usepackage[margin=2.2cm]{geometry}

\title{Theory Assignment-1: ADA Winter-2024}
\author{Rachit Arora (2022384) \and Mahi Mann (2022272)}

\date{}
\begin{document}

\maketitle

\section{Assumptions}

\begin{enumerate}
    \item $k$ ranges from $1$ to $3n$.
    \item $A \cup B \cup C$ is the concatenation of the 3 arrays $A$, $B$ and $C$. Formally, if $A = \{ a_1 , a_2 , \cdots a_n \}$, $B = \{ b_1 , b_2 , \cdots b_n \}$ and $C = \{ c_1 , c_2 \cdots c_n \}$; $A \cup B \cup C = \{ a_1 , a_2 , \cdots a_n , b_1 , b_2 , \cdots b_n , c_1 , c_2 , \cdots c_n\}$.   
    \item Array indices are from $1$ to $n$.
\end{enumerate}

\section{Preprocessing}
% Your preprocessing steps go here
$A$, $B$, $C$ and $n$ (the size of any one of the 3 sorted arrays) must already be stored in memory before the algorithm runs. 

\section{Algorithm Description}
% Describe your algorithm here

If a number is the $k \textsuperscript{th}$ smallest number in an array, it is greater than or equal to at least $k-1$ elements of the array. 

Note that if $a \geq b$ for any two real numbers $a$ and $b$, if $a$ is larger than or equal to $k_1$ numbers in the 3 arrays and $b$ is larger than or equal $k_2$ numbers, then $k_1 \geq k_2$. 

Since the arrays are sorted in increasing order, on each array, we can use binary search to calculate which number comes closest to $k-1$ numbers smaller than it. 

For each iteration of the binary search, we will lower-bound and upper-bound binary search the current value in all 3 arrays to calculate 

After repeating the search in all 3 arrays, we are sure to find the required number since at least one solution is guaranteed to exist.




\section{Recurrence Relation}
% Provide the recursive relation of your algorithm
The algorithm is not recursive.

\section{Complexity Analysis}
% Analyze the time and space complexity of your algorithm
$O(\log \textsuperscript{2} n)$

\section{Pseudocode}

Below is the pseudocode of the main calling function, 'KthSmallest'. Other functions called are also defined below the pseudocode of 'KthSmallest'.

\begin{algorithm}[H]
\begin{algorithmic}[1]

\caption{KthSmallest - O($\log \textsuperscript{2} n$)}


\Function{KthSmallest}{$\text{k}$}
    % Pseudocode for your algorithm
    \State $(v_1, b_1) \gets \text{FindClosest(A, N, k)}$
    \If {$b_1 = \text{True}$}
        \Return $v_1$
    \EndIf
    \State $(v_2, b_2) \gets \text{FindClosest(B, N, k)}$
    \If {$b_2 = \text{True}$}
        \Return $v_2$
    \EndIf
    \State $(v_3, b_3) \gets \text{FindClosest(C, N, k)}$
    \\ \phantom a \phantom . \Return $v_3$
\EndFunction

\end{algorithmic}
\end{algorithm}

The 'FindClosest' function is defined below: 

\begin{algorithm}[H]
\begin{algorithmic}[1]

\caption{FindClosest - O($\log \textsuperscript{2} N$)}

\Function{FindClosest}{$\text{Arr, N, Val}$}
    \State $l \gets 1$ 
    \State $r \gets N$ 
    \While {$l \leq r$}
        \State $m \gets \lfloor \frac{l+r}{2} \rfloor$
        \State $n_{gt} \gets \text{LowerBound(A, $m$, Arr[m])} + \text{LowerBound(B, $m$, Arr[m])} + \text{LowerBound(C, $m$, Arr[m])} - 3$
        \State $n_{geq} \gets \text{UpperBound(A, $m$, Arr[m])} + \text{UpperBound(B, $m$, Arr[m])} + \text{UpperBound(C, $m$, Arr[m])} - 3$
        \If {$n_{geq} \geq \text{Val} - 1$}
            \If {$n_{gt} \leq \text{Val} - 1$}
                \Return Arr[m]
            \Else
                \State $r \gets m-1$
            \EndIf
        \Else
            \State $l \gets m+1$
        \EndIf
    \EndWhile \\
    \phantom a \phantom .
    \Return $l$
\EndFunction

\end{algorithmic}
\end{algorithm}

The binary search functions 'LowerBound' and 'UpperBound' are also defined below:

\begin{algorithm}[H]
\begin{algorithmic}[1]

    \caption{LowerBound - O($\log N$)}

\Function{LowerBound}{$\text{Arr, N, Val}$}
    \State $l \gets 1$ 
    \State $r \gets N$ 
    \While {$l \leq r$}
        \State $m \gets \lfloor \frac{l+r}{2} \rfloor$
        \If {$\text{Arr}[m] \geq \text{Val}$}
            \State $r \gets m-1$
        \Else
            \State $l \gets m+1$
        \EndIf
    \EndWhile \\
    \phantom a \phantom .
    \Return $l$
\EndFunction

\end{algorithmic}
\end{algorithm}

\begin{algorithm}[H]
\begin{algorithmic}[1]
    \caption{UpperBound - O($\log N$)}
\Function{UpperBound}{$\text{Arr, N, Val}$}
    \State $l \gets 1$ 
    \State $r \gets N$ 
    \While {$l \leq r$}
        \State $m \gets \lfloor \frac{l+r}{2} \rfloor$
        \If {$\text{Arr}[m] > \text{Val}$}
            \State $r \gets m-1$
        \Else
            \State $l \gets m+1$
        \EndIf
    \EndWhile \\
    \phantom a \phantom .
    \Return $l$
\EndFunction

\end{algorithmic}
\end{algorithm}

\section{Proof of Correctness}
% Provide the proof of correctness for your algorithm


\end{document}
