\documentclass{article}
\usepackage[utf8]{inputenc}
\usepackage{amsmath}
\usepackage{algorithm}
\usepackage{algcompatible}
\usepackage{algpseudocode}
\usepackage{parskip}
\usepackage{commath}
\usepackage{hyperref}

\usepackage[margin=2.2cm]{geometry}

\title{Theory Assignment-1: ADA Winter-2024}
\author{Rachit Arora (2022384) \and Mahi Mann (2022272)}

\date{}
\begin{document}

\maketitle

\section{Assumptions}

\begin{enumerate}
    \item $k$ ranges from $1$ to $3n$.
    \item \textbf{Array indices are from $0$ to $n-1$.}
    \item \textbf{Passing in arrays or subarrays of an array into a function is a constant time operation, because in practice we can implement this by using start and end pointers instead. Arrays are passed directly in the pseudocode to avoid overly messy code.}
    \item For an array $A$, $\abs{A}$ is the size of array $A$. Similar to the previous point, accessing the size of an array is also a constant-time operation.
\end{enumerate}

\section{Preprocessing}
% Your preprocessing steps go here

None required.

\section{Algorithm Description}

We start with the arrays $A$, $B$ and $C$ as the main problem.

The algorithm is recursive. Consider the current subarrays to be $S_0$, $S_1$ and $S_3$, and the current element to be found as the $k\textsuperscript{th}$ smallest.

Define $mid(S_i) = \lfloor \frac{n_i}{2} \rfloor$ where $n_i$ is the size of the array $S_i$, where $i = 0,1,2$.  

We have two cases, $\sum_{i=0}^{2} mid(S_i) < k$ and $\sum_{i=0}^{2} mid(S_i) \geq k$. 

\begin{enumerate}

\item Let's consider the first case, $\sum_{i=0}^{2} mid(S_i) < k$.

    Without loss of generality, let $S_0[mid(S_0)]$ be the \textbf{minimum} of all $S_i[mid(S_i)]$ for $i=0,1,2$.

        Then, all elements of $S_0[0, \cdots, mid(S_0)]$ can be written as the $j\textsuperscript{th}$ smallest in $S_0 \cup S_1 \cup S_2$ where $j < k$ (refer to \hyperlink{section.7}{Proof of Correctness}).

        So, our required element may not be in $S_0[0, \cdots, mid(S_0)]$ because they can be at most the $(k-1)\textsuperscript{th}$ smallest, but it is definitely elsewhere in the remaining subarray and 2 arrays.

        So, we recursively find the $(k - mid(S_0))\textsuperscript{th}$ smallest element in $S_0[mid(S_0) + 1, \cdots, n_0 - 1] \cup S_1 \cup S_2$.

\item Now for the latter case, $\sum_{i=0}^{2} mid(S_i) \geq k$.

    Without loss of generality, let $S_0[mid(S_0)]$ be the \textbf{maximum} of all $S_i[mid(S_i)]$ for $i=0,1,2$.

        Then, all elements of $S_0[mid(S_0) + 1, \cdots, n_0 - 1]$ can be written as the $j\textsuperscript{th}$ smallest in $S_0 \cup S_1 \cup S_2$ where $j > k$ (refer to \hyperlink{section.7}{Proof of Correctness}).

        So, our required element may not be in $S_0[mid(S_0) + 1, \cdots, n_0 - 1]$ because they can be at least the $(k+1)\textsuperscript{th}$ smallest, but it is definitely elsewhere in the remaining subarray and 2 arrays.

        So, we recursively find the $k\textsuperscript{th}$ smallest element in $S_0[0, \cdots, mid(S_0)] \cup S_1 \cup S_2$.

\end{enumerate}

The base case for the recursion is simple. If the arrays $S_0$, $S_1$ and $S_2$ all have a single element, then the elements smaller than them will be $k-1$ in total, so we return the minimum of these 3 elements.

Furthermore, for completeness, if at any point of the algorithm an array cannot be divided further, we only recurse further on arrays which can.


\section{Recurrence Relation}
% Provide the recursive relation of your algorithm

Let $p = n_1n_2n_3$ be the current product of the sizes of A, B and C; $n_1$, $n_2$ and $n_3$ respectively. Since the size of one of the arrays in the next subproblem will be approximately halved, size of the next subproblem $p' = \frac{p}{2}$. The only serial (non-recursive) operation is comparing two elements in $A \cup B \cup C$, which can be done in $O(1)$ as defined, so the serial time taken is less than equal to $c_0$ for some positive real constant $c_0$. 

Therefore, we have:

\begin{align*}
    T(p) = T(p') + c_0 \\
    T(p) = T(\frac{p}{2}) + c_0
\end{align*}

\section{Complexity Analysis}
% Analyze the time and space complexity of your algorithm

The above recurrence relation can be expressed as 

\begin{align*}
    T(p) = aT(\frac{p}{b}) + k_0p^c
\end{align*}

where $a = 1, b = 2, c = 0$.

By Master's theorem, since $0 = \log_2{1}$ and thus $c = \log_b{a}$, 

\begin{align*}
    T(p) = O(\log p)
\end{align*}

Since $p = n_1n_2n_3$, $\log p = \log n_1 + \log n_2 + \log n_3 = 3 \log n$.

\textbf{Therefore, the overall problem with 3 arrays of size $n$ can be solved in $O(3 \log n) = O(\log n)$ time.} 

\section{Pseudocode}

\newpage

\begin{algorithm}[H]
\caption{KthSmallest}
\begin{algorithmic}[1]

\Function {KthSmallest}{A, B, C, n, k}
\\ \phantom a \phantom .
    \Return KthIndex(A, B, C, k-1)
\EndFunction

\\

\Function {KthIndex}{A, B, C, k} 
\\
    \If {A is empty and B is empty}
        \Return C[k]
    \EndIf
\\
    \If {A is empty and C is empty}
        \Return B[k]
    \EndIf
\\
    \If {C is empty and B is empty}
        \Return A[k]
    \EndIf
\\

    \For {$I \in \{A,B,C\}$}
    \If {I is empty}
        \State $m_I \gets \infty$
        \State $M_I \gets -\infty$
    \Else
        \State $m_I \gets$ I[Mid(I)] 
        \State $M_I \gets$ I[Mid(I)] 
    \EndIf
    \EndFor

\\
    \State $M \gets \max(M_A, M_B, M_C)$
    \State $m \gets \min(m_A, m_B, m_C)$

\\
    \State $n_{tot} \gets$ Mid(A) + Mid(B) + Mid(C)

\\

    \If {$n_{tot} < k$}
    \\
        \If {$m_A = m$}
        \\ \phantom a \phantom a \phantom a \phantom a \phantom . \phantom .
        \Return KthIndex(A[Mid(A)+1, $\cdots$, $\abs{A} - 1$], B, C, k - Mid(A) - 1)
        \ElsIf {$m_B = m$}
        \\ \phantom a \phantom a \phantom a \phantom a \phantom . \phantom .
        \Return KthIndex(A, B[Mid(B)+1, $\cdots, \abs{B} - 1$], C, k - Mid(B) - 1)
        \Else
        \\ \phantom a \phantom a \phantom a \phantom a \phantom . \phantom .
        \Return KthIndex(A, B, C[Mid(C)+1, $\cdots, \abs{C} - 1$], k - Mid(C) - 1)
        \EndIf
        \\
    \Else
    \\
        \If {$M_A = M$}
        \\ \phantom a \phantom a \phantom a \phantom a \phantom . \phantom .
        \Return KthIndex(A[$0, \cdots,$ Mid(A)], B, C, k)
        \ElsIf {$M_B = M$}
        \\ \phantom a \phantom a \phantom a \phantom a \phantom . \phantom .
        \Return KthIndex(A, B[0, $\cdots,$ Mid(B)], C, k)
        \Else
        \\ \phantom a \phantom a \phantom a \phantom a \phantom . \phantom .
        \Return KthIndex(A, B, C[0, $\cdots,$ Mid(C)], k)
        \EndIf
        \\
    \EndIf
\\
\EndFunction
\end{algorithmic}
\end{algorithm}

The Mid() function is defined as below:

\begin{algorithm}[H]
\begin{algorithmic}[1]
\Function {Mid}{Arr}
    \State $s \gets \abs{Arr}$
\\
\phantom a \phantom .
    \Return $\lfloor \frac{s}{2} \rfloor$ 
\EndFunction


\end{algorithmic}
\end{algorithm}

\section{Proof of Correctness}
% Provide the proof of correctness for your algorithm

The basic explanation and proof of how the algorithm works is provided in the algorithm description section itself. 

Here are some of the more minute details:

\begin{itemize}
    \item For the proof of $j < k$, consider the opposite. If $j \geq k$, the middle elements are at least the $(k+1)\textsuperscript{th}$ smallest, and therefore our assumption is false.
    \item For the proof of $j > k$ in case 2, consider the opposite. If $j \leq k$, the middle elements are at most the $k\textsuperscript{th}$ smallest, and therefore our assumption is false.
\end{itemize}

\end{document}
