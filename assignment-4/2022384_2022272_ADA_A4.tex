\documentclass{article}
\usepackage[utf8]{inputenc}
\usepackage{amsmath}
\usepackage{algorithm}
\usepackage{import}
\usepackage{algcompatible}
\usepackage{algpseudocode}
\usepackage{parskip}
\usepackage{minted}
\usepackage{commath}
\usepackage{tikz}
\usepackage{hyperref}

\usepackage[margin=2.2cm]{geometry}

\title{Theory Assignment-4: ADA Winter-2024}
\author{Rachit Arora (2022384) \and Mahi Mann (2022272)}

\date{}
\begin{document}

\maketitle

\section{Assumptions}

\begin{itemize}
    \item It is assumed that the input graph is a Directed Acyclic Graph, and no detection for incorrect input is required.

    \item By definition, $s$ and $t$ are always \emph{s-t cut} vertices, so they are always included in the set of \emph{s-t cut vertices} if a path from $s$ to $t$ exists. 
\end{itemize}

\section{Preprocessing}
% Your preprocessing steps go here

2 empty adjacency lists are required 

\section{Algorithm Description}

\textbf{Note}: The graph is $G = (V,E)$, the start node is $s$ and the target node is $t$.

\begin{enumerate}
    \item First, since we are given a Directed Acyclic Graph, it makes sense to topologically sort the nodes. 

    The toposort will be done using the standard algorithm of sorting nodes by decreasing post-numbers in the DFS traversal.

    \item We will remove the nodes which were not reachable by node $s$ in the DFS that we perfomed (because every path from $s$ to $t$ starts from $s$, and since no path from $s$ to those nodes exists, they cannot occur on a path from $s$ to $t$ and therefore cannot be \emph{s-t cut vertices}. So, it is safe to remove them from the graph.) 

    (\textbf{Note}: this step removes all nodes before $s$ in the topologically sorted order as well, so we do not need to do that step in post-processing. Proved in the "Proof of Correctness" section.)

    \item Similarly, we remove the nodes such that there is no path from that node to node $t$, since for a node to be an \emph{s-t cut vertex}, at least one path from the node to $t$ has to exist. Thus, we can safely remove them from the graph. 

    (\textbf{Note}: this step removes all nodes after $t$ in the topologically sorted order as well, so we do not need to do that step in post-processing. Proved in the "Proof of Correctness" section.)

    We can do this by performing a DFS on $G^R$ starting from $t$. All nodes not traversed in this DFS will be removed from the graph.

    \textbf{At this point in the algorithm, every node in the graph is reachable from $s$, and every node can reach $t$, and therefore, there is a path going from $s$ to $t$ through every node.}
\end{enumerate}

\section{Recurrence Relation}

\section{Complexity Analysis}
% Analyze the time and space complexity of your algorithm
\subsection*{DFS Traversal}

\begin{itemize}
    \item The time complexity of DFS traversal depends on the number of edges in the graph.
    \item For each vertex, we visit all its adjacent vertices exactly once.
    \item Therefore, the time complexity of DFS traversal is \(O(m + n)\).
\end{itemize}

\subsection*{Reverse DFS Traversal}

\begin{itemize}
    \item Similar to DFS traversal, reverse DFS traversal also has a time complexity of \(O(m + n)\).
\end{itemize}

\subsection*{Main Function}

\begin{itemize}
    \item The initialization of arrays takes constant time, \(O(n)\).
    \item Inputting the number of vertices and edges (\(n\) and \(m\)) also takes constant time.
    \item Constructing the adjacency lists takes \(O(m)\) time as we iterate through each edge and add it to the adjacency lists.
    \item Performing DFS traversal and reverse DFS traversal on the graph takes \(O(m + n)\) time each.
    \item Reversing the topologically sorted list takes \(O(n)\) time.
    \item The loop to find the cut vertices iterates through the topologically sorted list and checks the incident edges, which takes \(O(m + n)\) time.
    \item Overall, the time complexity of the main function is \(O(m + n)\).
\end{itemize}


\newpage 
\section{Pseudocode}

\begin{algorithm}[H]
\caption{Finding Cut Vertices}
\begin{algorithmic}[1]
    \State Initialize arrays $vis1[N]$, $vis2[N]$, and $incident[N]$ with zeros
    \State Initialize an empty list $toposorted$
    \Function{DFS1}{$s$}
        \State $vis1[s] \gets 1$
        \For{each vertex $v$ in $adj[s]$}
            \If{$vis1[v] = 0$}
                \State \Call{DFS1}{$v$}
            \EndIf
            \State $incident[v] \gets incident[v] + 1$
        \EndFor
        \State Add $s$ to $toposorted$
    \EndFunction
    \Function{DFS2}{$t$}
        \State $vis2[t] \gets 1$
        \For{each vertex $v$ in $adjrev[t]$}
            \If{$vis2[v] = 0$}
                \State \Call{DFS2}{$v$}
            \EndIf
        \EndFor
    \EndFunction
    \State Initialize arrays $vis1$, $vis2$, and $incident$
    \State Input $n$, $m$, $s$, and $t$
    \State Input the edges and construct the graph
    \State \Call{DFS1}{$s$}
    \State \Call{DFS2}{$t$}
    \If{$vis1[t] = 0$}
        \State Output "No cut vertices"
        \State \textbf{return}
    \EndIf
    \State Reverse $toposorted$
    \State Initialize $cnt \gets 0$ and an empty list $ans$
    \For{each vertex $node$ in $toposorted$}
        \If{$vis2[node] = 0$}
            \State \textbf{continue}
        \EndIf
        \State $cnt \gets cnt - incident[node]$
        \If{$cnt = 0$}
            \State Add $node$ to $ans$
        \EndIf
        \For{each vertex $edge$ in $adj[node]$}
            \If{$vis2[edge]$}
                \State $cnt \gets cnt + 1$
            \EndIf
        \EndFor
    \EndFor
    \For{each vertex $item$ in $ans$}
        \State Output $item$
    \EndFor
\end{algorithmic}
\end{algorithm}

\section{Proof of Correctness}
% Provide the proof of correctness for your algorithm

\section*{Addendum 1: C++ implementation}

\inputminted{cpp}{stcut.cpp}

\section*{Addendum 2: Test cases}

\include{./fig1.tex}
\include{fig2}

\section*{Addendum 3: People discussed with}

Swapnil Panigrahi, Rohak Kansal and Sahil Gupta.

\end{document}
