\documentclass{article}
\usepackage[utf8]{inputenc}
\usepackage{amsmath}
\usepackage{amsfonts}
\usepackage{dirtytalk}
\usepackage{parskip}
\usepackage{verbatim}
\usepackage{listings}
\usepackage{minted}
\usepackage{algorithm}
\usepackage{algpseudocode}

\title{ADA-2024: Homework 2}
\author{Rachit Arora, 2022384 \cr Mahi Mann, 2022272}
\date{\today}

\begin{document}
\maketitle

\textbf{Note:} throughout this doc we'll be assuming that the array $A$ is indexed from $1$ to $n$.

\section{Subproblem Definition}


\begin{align*}
    \text{MC(i, op, streak)} &= \text{maximum score upto the i\textsuperscript{th} index when \emph{op} (op $\in \{\text{Ring, Ding}\}$)} \\ &\quad \text{ was said at the i\textsuperscript{th} index,
                             and the current streak of saying \emph{op} is \emph{streak}.}
\end{align*}

\section{Recurrence of the subproblem}

Let's consider the base cases first.

\begin{align*}
    \text{MC(i, op, streak)} &= (-\infty) &\text{ (if i $<$ streak)}  \\
    \text{MC(1, Ring, 1)} &= \text{ A[1]} \\ 
    \text{MC(1, Ding, 1)} &= \text{(-A[1])} \\ 
\end{align*} 

Note that we're not working with the absolute values of A[i] here. These base cases handle either sign, and so will the recurrence relations below.

Now we come to the actual recurrences:

\begin{align*}
    \text{MC(i, Ring, 1)} &= \text{A[i]} + \max(\{\text{ MC(i-1, Ding, s) } | \text{ s } \in \{1,2,3\}\}) \\
    \text{MC(i, Ding, 1)} &= \text{(-A[i])} + \max(\{\text{ MC(i-1, Ring, s) } | \text{ s } \in \{1,2,3\}\}) \\
    \text{MC(i, Ring, s)} &= \text{A[i]} + \text{MC(i-1, Ring, s-1) (s $\in \{2,3\}$)} \\ 
    \text{MC(i, Ding, s)} &= \text{(-A[i])} + \text{MC(i-1, Ding, s-1) (s $\in \{2,3\}$)} \\ 
\end{align*}

Essentially, if the streak is 1, then the streak of the opposite saying ended at the last index. And if the streak is 2 or 3, the last index was forced to have the same saying. 

An case-wise analysis for the border cases has been omitted for the sake of sanity. The actual implementation in code is much cleaner, please refer to pseudocode or C++ implementation.

\section{Subproblem that solves the actual problem}

Because of how we've defined the subproblem, there are 6 subproblems at index n. We simply take the maximum as follows:

\begin{align*}
    \max(\{\text{ MC(n, op, streak) \textbar op $\in \{\text{Ring, Ding}\}$, streak $\in \{1,2,3\}$ }\})
\end{align*}

\section{Algorithm Description}

\subsection{Idea}
Pretty much described above. We'll have a 3-dimensional array of size 6n with all the answers. We'll start with the base cases, and index 3 onwards, compute all subproblems in index order.

\subsection{Pseudocode}

\textbf{Here, we are assuming that $(-\infty)$ is an arbritrarily large negative integer bound}

(Next page)

\begin{algorithm}[H]
\caption{MaxChickens}
\begin{algorithmic}[1]

\Function{MaxChickens}{A,n}
    \State MC $\gets$ array of dimension $(n \times 2 \times 3)$
    \\
    \State Ring $\gets$ 1
    \State Ding $\gets$ 2
    \State MC[1][Ring][2] $\gets (-\infty)$ 
    \State MC[1][Ring][3] $\gets (-\infty)$ 
    \State MC[1][Ding][2] $\gets (-\infty)$ 
    \State MC[1][Ding][3] $\gets (-\infty)$ 
    \State MC[1][Ring][1] $\gets$ A[1]
    \State MC[1][Ding][1] $\gets$ (-A[1])
    \\
    \For{$i = 2, 3, \cdots, n$}
        \For{streak $= 1,2,3$}
            \State MC[i][Ring][streak] $\gets (-\infty)$
            \State MC[i][Ding][streak] $\gets (-\infty)$
            \If{streak = 0}
                \For{j $= 1,2,3$}
                    \If{MC[i-1][Ding][j] $\neq (-\infty)$}
                        \If {MC[i-1][Ding][j] + A[i] $>$ MC[i][Ring][1]}
                            \State MC[i][Ring][1] $\gets$ MC[i-1][Ding][j] + A[i]
                        \EndIf
                    \EndIf
                    \If{MC[i-1][Ring][j] $\neq (-\infty)$}
                        \If {MC[i-1][Ring][j] - A[i] $>$ MC[i][Ding][1]}
                            \State MC[i][Ding][1] $\gets$ MC[i-1][Ring][j] - A[i]
                        \EndIf
                    \EndIf
                \EndFor
            \Else
                \If{MC[i-1][Ring][streak-1] $\neq (-\infty)$} \\
                    \phantom a \phantom a \phantom a \phantom a \phantom a \phantom a \phantom a \phantom a \phantom a
                    MC[i][Ring][streak] $\gets$ MC[i-1][Ring][streak-1] + A[i]
                \EndIf
                \If{MC[i-1][Ding][streak-1] $\neq (-\infty)$} \\
                    \phantom a \phantom a \phantom a \phantom a \phantom a \phantom a \phantom a \phantom a \phantom a
                    MC[i][Ding][streak] $\gets$ MC[i-1][Ding][streak-1] - A[i]
                \EndIf
            \EndIf
        \EndFor
    \EndFor
    \State ans $\gets (-\infty)$ 
    \For{$i \in \{\text{Ring, Ding}\}$}
        \For{$s \in \{1,2,3\}$}
            ans $\gets \max$(ans, MC[n][$i$][$s$])
        \EndFor
    \EndFor
    \\
                    \phantom a \phantom . 
    \Return ans
\EndFunction

\end{algorithmic}
\end{algorithm}
\section{Running time}

For every index i, there are at most $c \geq 6$ constant-time addition and comparison operations (for some natural number $c$), so

\begin{align*}
    T(i) \leq T(i-1) + c
\end{align*}

\textbf{We will now prove that $T(n) = O(n)$ by induction on $n$.}

Base case: at $n=1$, two values were initialised, so $2 \leq c$ addition and comparison operations.

Inductive Hypothesis: $T(k) \leq ck$ for some $k \in \mathbb{Z}$.

Inductive Step:

\begin{align*}
    T(k+1) &\leq T(k) + c \\
           &\leq ck + c &\text{ (by Inductive Hypothesis)} \\
           &= O(k+1)
\end{align*}

\textbf{Therefore, our algorithm runs in $O(n)$ time (linear in $n$)}.


\section{Addendum 1: C++ implementation}
\inputminted{cpp}{/home/rachit/prog/a.cpp}
\section{Addendum 2: Tests}
I ran these tests on my C++ implementation:

\begin{verbatim}
Input:
8
1 2 3 4 -1 -2 -3 -4
Output:
14
Comments:
DING RING RING RING DING RING DING DING

Input: 
8
-1 -2 -3 -4 1 2 3 4
Output:
14
Comments:
Negative of an array gives the same answer as expected (invert the sayings)

Input: 
100000
1 1 1 1 .... (100000 ones)
Output: 
50000
Comments:
RING RING RING DING (repeat this pattern)
\end{verbatim}

Also, here's a timed test of size 200000 to demonstrate how fast this algorithm is:

\begin{minted}{sh}
$ g++ a.cpp
$ time python -c "t=200000;print(str(t)+' 1'*t)" | ./a.out
100000
python -c "t=200000;print(str(t)+' 1'*t)"  0.01s user 0.01s system 70% cpu 0.022 total
./a.out  0.02s user 0.00s system 69% cpu 0.034 total
\end{minted}


\end{document}


